\chapter{Introduction} \label{CH0}\label{Ch0}

\section{Our Intended Audience}

These are notes for an introductory course on linear logic, 
%written bynon-logicians, and
intended for non-logicians.  Specifically, the
notes are intended for linguists who would like to read more about
some of the applications that linear logic has found in their field,
but are put off by the logical background required.
Most papers on linguistic applications of linear logic presuppose
fairly extensive logical competence.
For the average theoretical or computational 
linguist, existing texts on linear logic such
as \cite{Troelstra} provide a dauntingly technical introduction to
this material. Our hope is to provide a more gentle and accessible
introduction, focusing on those aspects of linear logic of most direct
linguistic relevance.

As a prerequisite we assume some (perhaps fading) recollection of
a standard introductory course on Montague semantics.  
That is, a reasonable familiarity
with first-order predicate calculus, a rudimentary knowledge of the
lambda-calculus and type theory, and a basic idea of what logical
inference is about.  If you have an intuitive understanding what the
following mean: 
\begin{quote}
$\forall x.\; \lf{man}(x)\rightarrow \lf{mortal}(x), \;\;\;
\lf{man}(\lf{john}) \;\;\;\vdash \;\; \lf{mortal}(\lf{john})$

$\lambda x. \lf{see}(x,\lf{john})\;(\lf{fred}) \;\;\; \equiv_\beta \;\;\;
\lf{see}(\lf{fred},\lf{john})$
\end{quote}
then you probably meet the prerequisite.


Linguistic applications of linear logic have principally been in the
areas of:
\begin{itemize}
\item Categorial and type-logical grammar \cite{Moortgat,vanBenthem},
including work on parsing categorial grammars \cite{Morrill,Hepple}, and the 
compositional semantics of categorial grammars \cite{Morrill,Carpenter}
\item `Glue semantics', which to a first approximation is a version of
categorial semantics but without an associated categorial grammar
\cite{Dalrymple}
\item Resource-based reformulations of other grammatical theories,
such as
Minimalism \mycite{Retore,Stabler}, Lexical Functional Grammar 
\cite{Saraswat,Muskens}
and Tree Adjoining Grammar \cite{Abrusci}
\item There have also been applications to such AI issues as the frame 
problem \cite{White}, which have some linguistic relevance
\end{itemize}
\if false
These linguistic applications push linear logic down two paths
relatively unexplored by `mainstream' logicians: (i) dealing with
word-order issues in grammatical theory, and the investigation of
non-commutative and multi-modal linear logics, and (ii) accounting for
and managing the vast degree of syntactic and semantic ambiguity in
natural language, and the investigation of various tabular proof
techniques.
\fi


We reiterate that this course aims at giving the necessary {\em
logical} background for beginning to understand the linguistic
applications of linear logic.  That is, our focus will be slanted
more to logic than to linguistics.  Rather than attempt to
exhaustively describe all the linguistic applications listed above,
we will single out only two for more sustained description: categorial
grammar, and glue semantics.  Others will receive much
briefer mention.  This approach may alienate a possible secondary
audience for these notes, namely linear logicians who would like
to learn something about linguistics.  We hope that they will
nonetheless find some things to interest them.  But for our primary
audience, linguists who would like to learn something about linear
logic, we believe that the logical focus is likely to target the
region of maximum need.

\section{The Linguistic Appeals of Linear Logic}

Linear logic has often been branded as a {\em resource-conscious}
logic, or a logic of resources.  This has undoubtedly been a major
part of its linguistic attraction.  Resource usage is an appealing
metaphor for thinking about various linguistic issues.  For example,
how a string of words provides a sequence of resources that can
be consumed to construct a syntactic analysis of a sentence.  Or
how word meanings provide a collection of resources that can be
used to construct the meaning of a sentence.  Or how linguistic
context can make certain resources available, such as possible pronoun
antecedents, that can be used to flesh out the interpretations of
words like {\it he}, {\it she} or {\it it}.  Indeed, it was this view
of linguistic context as a consumable and updatable resource that
originally attracted the present authors to the possible
applications of linear logic.

But it would be a mistake to think that linear logic was originally
devised for the purpose of being a logic of resources, in the way
that e.g. tense logics were devised to be logics of time.  Much
of the initial motivation came from an altogether different direction:
the role of {\em proofs} in logic.  
As Girard puts it \mycite{Gir:ssll} ``linear logic comes from a proof-theoretic
analysis of usual logic.'' To the extent that linear logic is
a logic of resources, the resources in question are premises,
assumptions and conclusions as they are used in logical proofs.

In brief, there is a programme to elevate the status of proofs to first class
logical objects: instead of asking `when is a formula $A$ true', we ask
`what is a proof of $A$?'.  Following Frege's distinction between
sense and denotation, proofs are to constitute the senses of logical
formulas, whose denotations might be truth values.  But there is a problem
with viewing proofs as logical objects.  We do not have direct access
to proofs, only to syntactic representations of them, in the form of
derivations in some proof system (e.g. axiomatic, natural
deduction or sequent calculus).  As current proof theory stands,
syntactic derivations are a flawed means of accessing the underlying
proof objects.  The syntax can introduce spurious differences between
derivations that do not correspond to differences in the underlying
proofs; it can also mask differences that really are there.

Some of the impetus behind the development of linear logic was to look
more closely at proofs, and to obtain a more satisfactory way of
describing the underlying proof objects.  Indeed, one of the outcomes
of linear logic is a new way of representing derivations, proof nets,
that supposedly more accurately reflect the structure of underlying
proofs.

It is not immediately obvious, perhaps, that this focus on proofs
gives linear logic further linguistic appeal.  But it does.  We will
point to just one example (which we expand on below): {\em parsing as
deduction}.  Under this view, parsing a sentence amounts to performing
a logical proof in a system where the words of the sentence
provide the premises, and the grammar the rules of inference.  The
parse tree of the sentence corresponds to the proof tree of the
derivation.  Moreover, this parse/proof tree is an object semantic
significance: it can be used to construct the meaning of this
sentence.  From this perspective, it is natural to take proofs as
first class objects, and to want to distinguish underlying proofs from
the idiosyncracies of a particular way of representing derivations.

\if false
\section{Organization of Notes}

The next two sections give a little more background to the two
linguistic appeals of linear logic just mentioned, namely 
(1) resource sensitivity,
and (2) proofs as objects.  
Section~\ref{Ch0Conn} then gives a brief
and informal introduction to the connectives of linear logic.

Chapter~\ref{Ch1Proof} does not deal with linear logic at all.  It
provides a basic introduction to proof theory.  This background is
necessary, as most of the features of linear logic are best understood
by reference to proof theory.  

Chapter~\ref{Ch2LinLogic} 
describes linear logic.  We will have little to say about the
{\em semantics} of linear logic, because of (a) space, (b) the fact
that the various forms of semantics proposed are for the
mathematically very sophisticated, and (c) it is of little linguistic
relevance.  

Chapter~\ref{Ch3ProofNets}  describes proof nets.  These are a new way
of doing and representing proofs in linear logic.  Proof nets have
been widely applied in catgeorial grammar.

Chapter~\ref{Ch4CatGram} describes the applications of linear logic
to categorial grammar, and discusses extensions to non-commutative
and multi-modal linear logic.

Chapter~\ref{Ch5Glue} describes the application of linear logic
to glue semantics.  This uses a modest fragment of (commutative)
linear logic to piece together the meanings of words and phrases
in a syntactically analysed sentence.

Chapter~\ref{Ch6Conclusion} discusses some other linguistic
applications of linear logic, and concludes.
\fi

\section{Resource Counting}

In traditional logics 
formulas denote truths or facts.  These facts may be used as little or
as often as one likes.  Take some simple facts about integers
\begin{quote}
$5>4, \;\;\; 4>3$\\
$1>0, \;\;\; 2>0$\\
$\forall i,j,k.\; (i>0 \;\wedge\; j>k) \;\rightarrow \;
          (i\times j)>(i\times k)$
\end{quote}
From these facts we can readily conclude, amongst other things, that
\begin{quote}
$(1\times 5)>(1\times 4) \;\wedge\; (2\times 5)>(2\times 4)$
\end{quote}
Reaching this conclusion makes one use apiece of the facts that $1>0$ and 
$2>0$, and two uses apiece of $5>4$ and the universally quantified fact.
It makes no use at all of the fact that $4>3$.  The truth of $5>4$ does
not wear out with repeated use: it remains just as true however many times
we make use of it.  Likewise, the truth of $4>3$ does not diminish into
falsehood through lack of use: it remains just as true however
few times we make use of it.

\bigskip

In linear logic formulas denote resources.  Resources, unlike truths,
get used up.  A stock example of resource consumption is: (a) a packet
of Gauloises costs 20FF, (b) a packet of Gitanes costs 20FF, therefore
(c) if I have (a resource of) 20FF I can either buy a packet of
Gauloises, or buy a packet of Gitanes, but not both.

Resource usage occurs in natural language at a very fundamental
level.  To a first approximation, each word and phrase in sentence is 
a resource that must get used exactly once in building a sentence.  If
a word needs to be used twice in building a sentence, it will occur in it
twice\footnote{Coordination is an exception to this crudely stated
generalization. The sentence ``{\em John ate and drank}'' bears an
equivalence to ``{\em John ate and John drank}'', where the word
``{\em John}'' occurs and is used twice.  It is these
occasional violations of a uniform `use-exactly-once' regime that 
lends linear logic much of its interest.  Linear logic, as we will see,
permits quite fine-grained control over regimes for resource usage.}.
And if a word is not to be used in building a sentence, it should not
occur in it.

As a more concrete example, consider an ambiguous sentence like
\begin{quote}
{\it John saw a man with a telescope.}
\end{quote}
where the prepositional phrase (PP) ``{\it with a telescope}'' can either
modify the noun phrase (NP) ``{\it a man}'' (so that the man has the
telescope), or the verb phrase (VP) ``{it saw a man}'' (so that John
is looking through the telescope).  We can picture this state of
affairs as follows
\begin{center}
\small
\begin{tabular}{cccccccc}
               & \node{a}{S} &              &               &\hspace*{3em} &
 & & \\[2ex]
\node{b}{NP}   &             & \node{c}{VP} &               & &
 &\node{j}{PP} & \\[2ex]
\node{d}{John} & \node{e}{V} &              & \node{f}{NP}  & &
\node{k}{P} & &\node{l}{NP} \\[2ex]
               & \node{g}{saw}&             &\node{i}{a man}& &
\node{m}{with}& &\node{n}{a telescope}
\end{tabular} 
\nodeconnect{a}{b}
\nodeconnect{a}{c}
\nodeconnect{b}{d}
\nodeconnect{c}{e}
\nodeconnect{c}{f}
\nodeconnect{e}{g}
\nodetriangle{f}{i}

\nodeconnect{j}{k}
\nodeconnect{j}{l}
\nodeconnect{k}{m}
\nodetriangle{l}{n}

{\makedash{4pt}
\anodecurve[t]{j}[r]{f}{10pt}[50pt]
\anodecurve[t]{j}[r]{c}{10pt}[50pt]
}
\end{center}
Here we have the phrase ``{\it with a telescope}'', which can attach
either to an adjacent VP or to an adjacent NP.  The crucial point is
that it has to be one or the other, but not both.  If the PP attaches
to the NP, it gets used up and cannot be used a second time to attach
to the VP; and vice versa.  The sentence cannot be interpreted as
saying that John looked through a telescope and saw a man holding a
telescope.

\bigskip

As a slightly more formal illustration of the resource differences
between traditional and linear logic, we can compare some valid and
invalid patterns of inference.  We will use the symbols $\rightarrow$
and $\wedge$ for traditional implication and conjunction, and
$\linimp$ and $\tensor$ for linear implication  and (multiplicative)
conjunction.

First, premises get consumed in linear logic in a way that they don't
in traditional logic:
\begin{quote}
\begin{tabbing}
Traditional implication: \=
             $A, A\rightarrow B \; \vdash \; B$\\
          \> $A, A\rightarrow B \; \vdash \; A\wedge B$
 \hspace*{5em}\= {\small $A$ is still true}\\[1ex]
 
Linear implication: \>
                     $A, A\linimp B \; \vdash B$\\
                    \> $A, A\linimp B \; \not\vdash A\tensor B$ 
                      \> {\small $A$ is used up}
\end{tabbing}
\end{quote}
In combining the two premises $A$ and $A\linimp B$ to derive $B$, both
the premises are used up.  This means that there is no longer an $A$
(nor an $A\linimp B$) around to be conjoined with the $B$.  But
in traditional logic, both premises are still available for re-use.

Unlike traditional logic, premises can't be ignored in linear
logic
\begin{quote}
\begin{tabbing}
Traditional conjunction: \= $A\wedge B \; \vdash \;
A$ \hspace*{6.5em}\={\small Can ignore $B$}\\[1ex]
Linear conjunction:\> $A\tensor B \; \not\vdash A$
                     \>{\small Have to use up $B$}
\end{tabbing}
\end{quote}
In linear logic, if you have an $A$ and a $B$ resource, you cannot
just throw one of them away; to get rid of it, you have to use it up.
In traditional logic, you can always choose to ignore some truths.


We will meet further linear logic connectives besides $\linimp$ and
$\tensor$ shortly, though the logical fragment defined by just these
two is already of considerable linguistic importance.  But one other
connective that is worth mentioning early is the exponential or modality !
(variously pronounced: `bang', `pling', `shriek', or `of course').  
The modality
allows repeated use or discarding of any formula/resource to which it
applies
\begin{quote}
\begin{tabbing}
Of course: \= $!A \; \vdash \; A\tensor !A$ \hspace*{5em}\={\small Re-use}\\
           \> $!(A)\tensor B \; \vdash \; B$ \>{\small Discard}
\end{tabbing}
\end{quote}
Judicious use of this modality allows finer-grained control over
resource sensitivity.  At one extreme, by banging every formula you
get traditional, non-resource sensitive logic.


\section{Proofs as Objects}

\subsection{Structural Rules and Premise Counting}

Linear logic, and its inherent resource sensitivity, arises from
from consideration of the way that premises are used in
proofs\footnote{Because of this it has recently been argued \mycite{Pym}
that linear logic is a fairly impoverished logic of resources.  It
works well for premise counting, but not so well for more naturally
occurring, divisible resources like time or money.  Whatever the
merits of this argument, the kind of resource
sensitivity corresponding to premise counting is linguistically the
most useful.}.  Traditionally, proofs are from (sub)sets of premises.
The identity of a set is not changed if elements in it are repeated:
hence premises can be re-used.  If a conclusion follows from a certain
set of premises, further premises can be added without invalidating
the conclusion. Likewise, premises not used in a proof can safely
be removed: premises can be spirited out of thin air, and discarded in
the same way.  Finally, the order of the premises in the set is
immaterial to the identity of the set.

These observations about sets of premises point two three {\em
structural} rules of inference at work in traditional logic.
The first is contraction\footnote{For now, we are being fairly
informal and sloppy in the way we present these rules, and the
intuitive explanations are more important.  More rigour will ensue
when we discuss proof systems properly.}:
\begin{quote}
\begin{prooftree}
\Gamma, \; A, \; A \; \vdash \; B
\justifies
\Gamma, \; A \;  \vdash \; B
\using {\small contraction}
\end{prooftree}
\end{quote}
This says that if $B$ follows from $\Gamma$ plus two uses of $A$, it
follows from $\Gamma$ plus a single occurrence of $A$.  We can always
duplicate the $A$ to get the second occurrence.

The second structural rule is weakening:
\begin{quote}
\begin{prooftree}
\Gamma \; \vdash \; B
\justifies
\Gamma, \; A \;  \vdash \; B
\using {\small weakening}
\end{prooftree}
\end{quote}
This says that if $B$ follows from $\Gamma$, it does no harm to
expand, stretch or weaken the set of premises to include $A$.

The third structural rule is exchange:
\begin{quote}
\begin{prooftree}
\Gamma, \; A, \; B \; \vdash \; C
\justifies
\Gamma, \; B, \; A\;  \vdash \; C
\using {\small exchange}
\end{prooftree}
\end{quote}
This says that the order in which the premises are presented does not matter.


Linear logic results from dropping the rules of contraction and
weakening, so that premise counting becomes important.  
Rather than (sub)sets of premises, linear logic operates on {\em
multisets} of premises. Dropping the two structural rules leads
directly to the need to define new linear connectives, though we will
defer discussion of this until a later chapter.

Linear logic preserves
the rule of exchange however, so that premise order remains unimportant.
For
grammatical applications, premise order often corresponds to word order
in a sentence, and is very important.  Non-commutative linear logics
result from modifying the exchange rule to account for order
phenomena.

\subsection{Proof Structures}

The development of linear logic was also, in part, motivated by a 
deep interest in the structure of proofs.  A major branch of
proof theory is concerned not with just establishing whether
a conclusion follows from its premises, but with the form any such proof
takes.  Independent of it specific application to linear logic, this
concern with proof structure is of major significance to linguists;
it should be particularly familiar to those operating within the 
paradigm of `parsing as deduction'.  We will therefore spend some time
introducing the basic ideas for traditional (i.e. non-linear) logic,
before briefly bringing the discussion back to linear logic.

\paragraph{Parsing as Deduction}
A version of (context free) parsing as deduction holds that grammar
rules like
\begin{quote}
\begin{tabular}{llll}
S & $\Rightarrow$ & NP & VP\\
VP & $\Rightarrow$ & V & NP\\
PP & $\Rightarrow$ & P & NP\\
NP & $\Rightarrow$ & NP & PP\\
VP & $\Rightarrow$ & VP & PP\\
NP & $\Rightarrow$ & Det & N
\end{tabular}
\end{quote}
should be regarded as logical implications in reverse; for example the
existence of an adjacent Noun Phrase and Verb Phrase implies the
existence of a Sentence spanning them both.  Marking parameterized
string positions ($i$, $j$, $k$) on the rules, we can rewrite them as logical
implications
\begin{quote}
\begin{tabular}{lllll}
$_i$NP$_j$ &$\wedge$& $_j$VP$_k$ & $\rightarrow$ & $_i$S$_k$\\
$_i$V$_j$ &$\wedge$& $_j$VP$_k$ & $\rightarrow$ & $_i$VP$_k$\\
$_i$P$_j$ &$\wedge$& $_j$NP$_k$ & $\rightarrow$ & $_i$PP$_k$\\
$_i$NP$_j$ &$\wedge$& $_j$PP$_k$ & $\rightarrow$ & $_i$NP$_k$\\
$_i$VP$_j$ &$\wedge$& $_j$PP$_k$ & $\rightarrow$ & $_i$VP$_k$\\
$_i$Det$_j$ &$\wedge$& $_j$N$_k$ & $\rightarrow$ & $_i$NP$_k$
\end{tabular}
\end{quote}
The first rule says that an NP from position $i-j$ and a VP from
position $j-k$ implies a sentence from position $i-k$.
A sentence like ``{\it John saw a man with a telescope}'' gives
rise to a set of lexical premises
\begin{quote}
\begin{tabbing}
$_0$NP$_1$, \ \ \= $_1$V$_2$,\ \  \= $_2$Det$_3$, \= \ldots, \ \ \= $_6$N$_7$\\
John        \> saw        \> a           \> \ldots  \> telescope
\end{tabbing}
\end{quote}
Letting $\Gamma$ be the lexical premises plus the grammar rules,
parsing becomes a search for a proof that
$\Gamma \; \vdash \; _0\mbox{S}_7$.
What is interesting is not that $_0$S$_7$ follows from $\Gamma$ so
much as there are two quite different ways of proving it.  These two
proofs correspond to the readings where (i) the PP
``{\it with a telescope}'' attaches to the NP ``{\it a man}'', and
(ii) the PP attaches to the VP ``{\it saw a man}''\footnote{The alert
reader will have realized that without any restriction on resource
usage there is a third proof: where the PP attaches to {\em both} the
NP {\em and}
the VP.  Implementations of parsing as deduction, such as Direct
Clause Grammars \mycite{???}, usually include implicit resource control
in the inference engine to prevent this.  Using linear logic makes
resource issues explicit in the (logicized) grammar, rather than
implicit in its interpreter.}.  The bald statement that $_0$S$_7$
follows from $\Gamma$ does not reveal the existence of the two parses.
Studying the structures of the proofs, on the other hand, does reveal
the number of parses.  Moreover, these differences in proof structure
give rise to differences in meaning when we come to apply some form of
compositional semantic interpretation.

More generally, for parsing as deduction proof structures {\em are}
syntactic structures.  A phrase structure tree can be viewed as a
proof tree (though by convention, logicians write their proof trees
upside down). For example
\begin{center}
\begin{tabular}{ll}
\begin{tabular}{ccccc}
               & \node{a}{S} &              &             &  \\[2ex]
\node{b}{NP}   &             & \node{c}{VP} &             &  \\[2ex]
\node{d}{John} & \node{e}{V} &              & \node{f}{NP}&  \\[2ex]
               & \node{g}{saw}&\node{h}{Det}&             &\node{i}{N}\\[2ex]
               &              &\node{j}{a}  &             &\node{k}{man}
\end{tabular} 
\nodeconnect{a}{b}
\nodeconnect{a}{c}
\nodeconnect{b}{d}
\nodeconnect{c}{e}
\nodeconnect{c}{f}
\nodeconnect{e}{g}
\nodeconnect{f}{h}
\nodeconnect{f}{i}
\nodeconnect{h}{j}
\nodeconnect{i}{k}

= 
& \hspace*{1em}
\begin{prooftree}
\mbox{NP}\hspace*{1em}
  \[\mbox{V} \hspace*{1em}
     \[\mbox{Det} \hspace*{1em} \mbox{N}
       \justifies \mbox{NP}
       \using \mbox{\tiny Det$\wedge$N$\rightarrow$NP}
     \]
     \justifies \mbox{VP}
     \using \mbox{\tiny V$\wedge$NP$\rightarrow$VP}
  \]
  \justifies \mbox{S}
  \using \mbox{\tiny NP$\wedge$VP$\rightarrow$S}
\end{prooftree}

\end{tabular}
\end{center}
Furthermore, these structures carry semantically relevant information.
In fact, proof structures have a non-trivial semantics, expressible
by means of the lambda-calculus.  This is not without significance for
linguists working on the semantics of natural language.


\paragraph{The Semantics of Proofs --- the Curry-Howard Isomorphism:}
Within proof theory, the `semantics' of proof structures is a topic of
major interest.  It turns out that superficially distinct proofs can
sometimes be semantically equivalent.  These distinct but equivalent
proofs can often be grouped together into an equivalence class,
identifiable by a canonical form of the proof.  Purely
syntactic operations on any non-canonical proof, such as cut-elimination
or proof normalization, can convert it to the canonical form.  At a
semantic level, these conversions amount to performing various types of
lambda-reduction on the semantics of the proofs.  

Although we will discuss the intimate connection between proofs,
type-theory and the lambda calculus in more detail in chapter~\ref{CH1},
a preliminary illustration is in order.  Consider the standard
natural deduction elimination and introduction rules for implication:
\begin{center}
\begin{tabular}{ccc}
\begin{prooftree}
A\imp B \hspace*{3em} A
\justifies B
\using \impE
\end{prooftree}
& \hspace*{5em}
\begin{prooftree}
\[[A]^i\resultsin B\]
\justifies  A\imp B
\using \impI,i
\end{prooftree}
\end{tabular}
\end{center}
The elimination rule $\impE$ is just {\it modus ponens}, from $A$ and $A\imp
B$, conclude $B$.  The introduction rule $\impI$ is read as follows. 
Assume that $A$, and label this assumption $i$ for
book-keeping purposes.  Suppose that from the assumption $A$ plus some
other premises, you can prove $B$.  Then you can discharge the
assumption of $A$ to instead conclude that if you were given $A$ as a
real premise you could derive $B$; i.e. conclude $A\imp B$. The box
around the assumption marks it as discharged, and the index on
the introduction rule shows which assumption is being discharged.  The
conclusion $A\imp B$ is no longer depedendent on the assumption $A^i$ in
the way that the intermediate conclusion $B$ is.


Both of these standard inference rules can be paired with semantic
operations on {\em (proof) terms}:
\begin{center}
\begin{tabular}{ccc}
\begin{prooftree}
f:A\imp B \hspace*{3em} a:A
\justifies f(a):B
\using \impE
\end{prooftree}
& \hspace*{5em}
\begin{prooftree}
\[[x:A]^i\resultsin f(x):B\]
\justifies  \lambda x.f(x):A\imp B
\using \impI,i
\end{prooftree}
\end{tabular}
\end{center}
We should think about the rule for {\it modus ponens} / $\impE$ as
follows:
An implication $A\imp B$ can be regarded as a function that takes
things of type $A$ and returns things of type $B$.  Let us call the
function $f$, and note that it will have a type $A\imp B$.  Suppose
that we have an object $a$ of type $A$.  Applying the function $f$ to
it will give us an object $f(a)$, which has type $B$.  Or put another
way, the semantics of modus ponens correspdonds to the {\em functional
application} of the implication to its (antecedent) argument.

The rule for implication introduction, $\impI$,
 corresponds to lambda-abstraction.  Assuming some
arbitrary $x$ of type $A$, let us suppose we can construct some object
$f(x)$ of type $B$.  Then we can abstract over the arbitrary $x$ to
create a function $\lambda x.f(x)$ that can take any object of type
$A$ as an argument, and return an object of type $B$.  That is, the
function $\lambda x.f(x)$ has type $A\imp B$.

The pairing of proof rules with ($\lambda$-calculus) operations on
proof terms is known as the {\em Curry-Howard Isomorphism}\footnote{The
isomorphism does not hold for all logics, and it does not hold for 
styles of proof system. The original version of the isomorphism was
discovered for intuitionistic (as opposed to classical) logic, and for
a natural deduction (as opposed to axiomatic) proof system.}  The term
labelling each formula can be seen as a description of how the formula is
derived/proved.  (Premises are usually labelled with arbitrary atomic terms).
Conversely, the formula, can be seen as giving the logical type of the proof.


Lambda-equivalences between proof terms can be used to show when two
superficially distinct proofs are essentially the same. As an example
here are two proofs that from $A$ and $A\imp B$ you can conclude $B$.
The first proof (\ref{proof1}\ref{simpleproof1}) is sensible,
 the second (\ref{proof1}\ref{complexproof1}) is pointlessly complicated:
\begin{ex}\label{proof1}
\begin{subexamples}
\item \label{simpleproof1}
\begin{prooftree}
A\imp B \hspace*{3em} A
\justifies B
\using \impE
\end{prooftree}

\bigskip

\item \label{complexproof1}
\begin{prooftree}
A \hspace*{3em}
\[ \[[A]^i \hspace*{3em} A\imp B \justifies B \using \impE\]
   \justifies A\imp B \using \impI,i
\]
\justifies   B
\using \impE
\end{prooftree}
\end{subexamples}
\end{ex}
By considering proof terms we can show that the second, more complex
derivation is an uninteresting variant of the first:
\begin{ex}\label{proof2}
\begin{subexamples}
\item \label{simpleproof2}
\begin{prooftree}
f:A\imp B \hspace*{3em} a:A
\justifies f(a):B
\using \impE
\end{prooftree}

\bigskip

\item \label{complexproof2}
\begin{prooftree}
a:A \hspace*{3em}
\[ \[[x:A]^i \hspace*{3em} f:A\imp B \justifies f(x):B \using \impE\]
   \justifies \lambda x.f(x):A\imp B \using \impI,i
\]
\justifies  (\lambda x.f(x))(a):B
\using \impE
\end{prooftree}
\end{subexamples}
\end{ex}
Given the standard lambda-calculus rules of $\beta$- and 
$\eta$-conversion\footnote{
These rules are:
\begin{itemize}
\item
$\beta$-conversion (lambda-reduction):
$(\lambda x.\phi)(a) \; = \; \phi[a/x]$\\
where $\phi[a/x]$ indicates substitution of $x$ by $a$ in $\phi$

\item 
$\eta$-conversion (extensionality):
$\lambda x.\phi(x) \; = \; \phi $\\
provided that $x$ does not occur in $\phi$
\end{itemize}
}
note how the proof terms for (\ref{proof2}\ref{simpleproof2}) and
(\ref{proof2}\ref{complexproof2}) are equivalent,
\[f(a) \; = \; (\lambda x.f(x))(a)\]
indicating a semantic equivalence between the two derivations.
Chapter~\ref{CH1} describes operations of {\it proof normalization},
corresponding to $\beta$- and 
$\eta$-conversion of proof terms, that reduce
derivations to their simplest canonical form.  These show, for
example, that (\ref{proof1}\ref{simpleproof1}) is the normal-form
version of derivation (\ref{proof1}\ref{complexproof1}).

To sum up, proofs structures are interesting in part because they have
non-trivial identity criteria.  In well-behaved logical systems, superificially
distinct proofs can be mapped onto a common, canonical form.  Moreover, 
terms can be assigned to formulas in a proof, embuing the proof with some
form of semantics.  Proof normalization operations are meaning-preserving
with regard to proofs.  The relevance of this for linguists is two-fold.
If proof trees can correspond to parse trees, the existence of canonical form
proofs suggests ways of limiting one's search for parse trees.  Second,
the semantics of proofs gives a handle on the semantics of parse trees.
That is, the Curry-Howard Isomorphism can be a major tool for natural language
semantics.  This is indeed precisely the trick employed by categorial grammar
as well as by glue semantics.

\subparagraph{Linear Logic \& Semantically Equivalent Proofs:}
Our discussion of the Curry-Howard Isomorphism, with its semantic 
identities between prooofs, has not touched on linear logic.
However, as mentioned in section~\ref{???}, the structure and
identity criteria for proofs is a major theme lying behind much work
on linear logic.  Not only can versions of the Curry-Howard
Isomorphism be extended to linear logic, but new ways of representing
certain types of proof are available --- proof nets.

\section{A Rough Guide to the Linear Connectives}

In this section we give an informal introduction to the connectives
and constants of linear logic, in the hope of providing the reader 
with an intuitive feel for their meanings.  This task is not as easy
as we would have liked.  Some of the connectives and constants of
linear logic are peculiarly resistant to informal explanation.
Where this is the case, we will refer the reader to
chapter~\ref{CH2} for explanation, and not attempt it here.
It should be borne in mind that the explanations given here are
meant to be highly informal, and solely for the purpose of providing
the reader with some initial familiarisation.  

One of the most immediately striking things about linear logic is that
it has two forms of conjunction ($\tensor$ and $\lwith$) and two forms
of disjunction ($\parr$ and $\lplus$).  To go with these, it also
has two forms of true ($\top$ and $\lone$), and two forms of false
($\bot$ and $\lzero$).  Fortunately, there is just one form of
implication, so we will start with this:
\begin{itemize}
\item Linear implication: $\linimp$\\
$A\linimp B$ means that can consume an $A$ resource to produce a $B$ resource.
\item Negation: $.^{\bot}$\\
$A^{\bot}$ roughly speaking stands for something that will consume an
$A$ resource.  Resources come paired, a little like matter and
anti-matter.  A production $A$ meets with a consumption $A^{\bot}$ to 
leave nothing at all.  Negation of a consumer gives rise to a
producer, and vice versa, so that $A^{\bot\bot}\equiv A$.
\item Tensor (multiplicative conjunction): $\tensor$\\
$A\tensor B$ means that your resources make both $A$ and  $B$
available.
If you have a resource $A\tensor B$, you can recover both $A$ and $B$
simultaneously.
\item Par (multiplicative disjunction): $\parr$\\
This is one of the connectives that is hardest to explain.  One way
of looking at it is merely to note that $A\linimp B$ can be defined
as $A^{\bot}\parr B$.  That is either you have something that is
looking to consume an $A$ resource, or you produce a $B$ resource.
One can try to paraphrase $A\parr B$ as `if you don't have an $A$ then
you have a $B$, and vice versa.'
\item With (additive conjunction): $\lwith$\\
$A\lwith B$ means that your resources can make $A$ available, and they can
make $B$ available, but not both simultaneously.  In terms of proofs, 
your premises
allow a proof establishing $A$, and they also allow a (separate) proof
establishing $B$.  But because proofs consume premises, you cannot put
both of these proofs together using just the one set of premises.\\
This is also sometimes known as {\em internal choice}: we can decide whether
to obtain $A$ or to obtain $B$.  
\item Plus (additive disjunction): $\lplus$\\
$A\lplus B$ means your resources make either $A$ or $B$ available, but you
don't know which.\\
This is also sometimes known as {\em external choice}
\item Of course: $!$\\
$!A$ means that you can produce as many copies of the $A$ resource as you like,
including zero copies.  
\item Why not: $?$\\
$?A$ means that you can consume as many copies of the $A$ resource as you like,
including zero copies.
\item Unit: $\lone$\\
This is the identity for tensor, so that 
$(A\tensor\lone)\equiv A$.  Unit is the trivial resource that can
be produced from nothing.  Another way of putting this is that if
a collection of resources produces $\lone$ (and nothing else), then we
can consume / throw away that collection of resources
\item Top: $\top$\\
This is the identity for with, so that $(A\lwith\top)\equiv A$.
Top consumes all resources
\item Imposibility: $\lzero$\\
This is the identity for plus, so that $(A\lplus\lzero) \equiv A$. It
corresonds to the impossible resource, so that an external choice
between $A$ and $\lzero$ must always result in $A$.  Note also that
$\lzero \equiv \top^{\bot}$.  Since $\top$ consumes all resources,
$\lzero$ produces all resources.  In this respect, it is like logical
falsehood, from which all possible conclusions follow.
\item Bottom: $\bot$\\
This is the identity for par, so that $(A\parr\bot)\equiv A$.  It is
also the dual of $\lone$, so that $\lone^{\bot}\equiv \bot$
\end{itemize}
This is a large collection of connectives.  The reader will be cheered
to know that for linguistic purposes implication and tensor are by far
and away the most significant.

It seems to have become a tradition to illustrate the meanings of some
of the connectives in relation to the interpretation of a fixed price
menu at a restaurant.  Not wishing to break the tradition:
\begin{center}
\begin{tabular}{ccc}
Menu: \pounds 5       & \hspace*{3em} & $(P\tensor P\tensor P\tensor P\tensor P)$\\
                      & & $\linimp$\\
Fish                  & & $[Fish$\\
                      & & $\tensor$\\
Chips                 & & $Chips$\\
                      & & $\tensor$\\
Soup or Salad         & & $(Soup \lwith Salad)$\\
                      & & $\tensor$\\
Fruit or cheese       & & $(Fruit \lplus Cheese)$\\ 
(depending on availability) & & \\
                      & & $\tensor$\\
Coffee                & & $Coffee$\\
(free refills)        & & $!Coffee]$ 
\end{tabular}
\end{center}
Here we begin with five one pound resources ($P$).  Note how $\lwith$
is used to mark the choice over which we have control (soup or salad),
whereas $\lplus$ is used to mark the choice over which the restaurant
has control (fruit or cheese).  The implication could also be
represented as
\[(P\tensor P\tensor P\tensor P\tensor P)^{\bot} \parr [Fish\tensor\ldots
\tensor !Coffee]\]
This says that either we are left with something that wants to consume
five pounds, or we are left with the meal.


